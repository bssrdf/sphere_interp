\documentclass[12pt,a4paper]{article}

%----------------Package imports-------------------
\usepackage{graphicx}
\usepackage{algpseudocode}
\usepackage{amssymb}
\usepackage{amsmath}

%------------Page Definition-----------------------
\usepackage[margin=1in]{geometry}

%----------------- SETUP PAGE HEADER ------------------------
\usepackage{fancyhdr}
\pagenumbering{arabic}
\pagestyle{fancy}

%----------------- DEFINE STUFF -----------------
\newcommand{\eps}{\varepsilon}
\newcommand{\lap}{\bigtriangleup}
\newcommand{\mean}{\mathrm{mean}}
\renewcommand{\algorithmicrequire}{\textbf{Input:}}
\renewcommand{\algorithmicensure}{\textbf{Output:}}

\begin{document}

%------------TITLE---------------
\thispagestyle{empty}
\begin{center}
\rule{\textwidth}{.5mm} \\[0.8cm]
{\huge \bfseries Smooth Spherical Interpolation\\ --Theory--}\\[.4cm]
\rule{\textwidth}{.5mm} \\[1.5cm]

{\Large \textsl{Author}\\[-.6cm]\rule{60pt}{.2mm} \\[.4cm] Peter M. Solfest}
\newpage
\end{center}
%------------End TITLE------------------------
\section{Introduction}
Given a sparse set of data points around a sphere,
it is convenient to have an analytical function fitting
the data which allows for which to interpolate to any point on the sphere.
Much work has been done representing a function in spherical harmonics,
which can then be used for such interpolations.

Techniques using exact quadrature exist, but
require data available on a regularly or gaussian spaced grid.
For sparse real world data, this spacing is often not available.
Thus techniques involving exact quadrature fail.
Furthermore, it is not clear that an inexact quadrature scheme can
be used given the arbitrary nature of real world locations.

Presented below is a method which fits arbitrary data to
spherical harmonics of specified degree, minimizing a function
involving terms measuring smoothness in addition to the least
squares fitting.
This smoothing not only allows for a determination of an
arbitrarily large set of spherical harmonics, but also acts
to minimize the large oscillations (gibbs phenomenon) introduced
by spherically fitting data which may exhibit wide variations.

Furthermore, the final algorithm can be computed
utilizing an iterative scheme involving only matvecs, saving
the computational expense of solving a least squares problem.

\section{Theory}
Let there be $s$ sample points, and $b$ the number of basis functions used.
Traditionally spherical harmonics are labeled $Y_{l,m}$, so an easy map for
indices is $i = l (l+1) + m$, so $Y_{l,m} = Y_i$.
Our data points will be labeled $f = f[j] = x_j$ for each sampled location $x_j$.

The goal is to find a set of coefficients $c_i$ such that at an arbitrary point $x$:
\[
    f(x) = \sum_{i=0}^{(l+1)^2} c_i \cdot Y_i(x)
\]
where $f(x_j) = f_j + \eps$ for small $\eps$ is desired.

In order to do this, we need a vandermonde like matrix, $M \in \mathbb{R}^ {s \times b}$, such that
\[
    M_{j,i} = Y_i(s_j)
\]

Potentially the most straightforward way to form the coefficients $c_i$ is by solving
the least squares problem
\[
    M c = f
\]
This is equivalent to solving\footnote{$\|\cdot\| = \|\cdot\|_2$}
\[
    \min_c \| M c - f \|
\]
The problem with this is $b$ can grow to infinity, creating increasingly overdetermined
systems.
Furthermore, if a bad data point at $x_i$ is near $x_j$, but $f_i$ is far from $x_j$, the
least squares solution will match these incredibly well, leading to large oscillations.

To prevent this, we can choose a better objective function to minimize, namely
\begin{equation}\label{cont_obj}
    \min_c \left(\frac{1}{2}\| \lap f \|^2 + \lambda \| M c - f \|^2\right)
\end{equation}
Where $\lambda \in [0,\infty)$ determines what we are interested in,
and the $\frac{1}{2}$ is introduced for convenience.
Note that $\lambda = 0$ corresponds to a perfectly smooth function, namely
the mean of $f_j$.
Whereas $\lambda = \infty$ corresponds to the least squares solution.

One of the beauties of this is that spherical harmonics are eigenfunctions
of the laplacians, so we get
\[
    \lambda f = B c
\]
where $B \in \mathbb{Z}^{b \times b}$ is a diagonal matrix 
such that $B_{i,i} = l_i^2 \cdot (l_i+1)^2$,
where $l_i$ is the $l$ corresponding to $Y_i$.

Thus equation \ref{cont_obj} yields equation \ref{obj_func} when discretized.
\begin{equation}\label{obj_func}
    \min_c \left( \frac{1}{2} c^t B c + \lambda (Mc - f)^T (Mc - f) \right)
\end{equation}
This is a quadratic equation, so differentiating and setting equal to zero yields
that the solution\footnote{Note that differentiating twice yields $B + 2 \lambda M^T M$ which
is positive definite. Thus it is indeed the minimum as the only solution.} is
\[
    B c + 2 \lambda M^T M c - 2 \lambda M^T f = 0
\]
Rearranging yields equation \ref{min_sol}
\begin{equation}\label{min_sol}
    \dfrac{1}{2\lambda} c = B^{-1} M^T f - B^{-1} M^T M c
\end{equation}

\section{Iteration}
This arrangement immediately suggests an iterative technique for solving this problem.
Namely,
\begin{equation}\label{iter_alg}
    c_i = 2\lambda \left( B^{-1} M^T f - B^{-1} M^T M c_{i-1} \right)
\end{equation}
Also, define $c \cdot e_1 = \mean(f_i)$.
This lets our iterations ignore the $l=0, m=0$ term and only add it back in
when reconstructing $f$.
Also, $B^{-1}$ is trivial to calculate since $B$ is diagonal.
Furthermore, the positive term on the right is constant, so only needs to
be computed once, leaving a mere 2 matvecs per iteration in the solution.
Thus a reasonable starting point is $c_0 = 0$, 
or equivalently $c_1 = 2 \lambda B^{-1} M^T f$.

\subsection{Convergence analysis}
First, consider
\begin{align}\label{iter_eqn}
    \| c_{i+1} - c_i\| &= \| 2 \lambda B^{-1} M^T f - 2 \lambda B^{-1} M^T M c_i - c_i\|\\
    &= \| 2 \lambda B^{-1}M^T f - (I + 2 \lambda B^{-1}) M^T M) c_i \|
\end{align}
Furthermore,
\begin{align}
\|c_{i+2} - c_{i+1}\| &= \| 2 \lambda B^{-1} M^T f - (I + 2 \lambda B^{-1} M^T M) c_{i+1} \| \\
&= \| 2 \lambda B^{-1} M^T f - (I + 2 \lambda B^{-1} M^T M) 2 \lambda (B^{-1} M^T f - B^{-1} M^T M c_i)\| \\
&=\| 2 \lambda (B^{-1}M^T f - B^{-1}M^Tf + B^{-1}M^TMc_i - 2\lambda B^{-1}M^TMB^{-1}M^Tf + \\
&  2\lambda B^{-1}M^TMB^{-1}M^TMc_i) \| \\
&\leq 2 \lambda \|B^{-1}M^TM\| \cdot \|2\lambda B^{-1}M^Tf - (I + 2\lambda B^{-1} M^TM)c_i\| \\
&\leq 2 \lambda \|B^{-1}M^TM\| \cdot \|c_{i+1} - c_i \|
\end{align}
Thus using that $\|B^{-1}\| = \dfrac{1}{4}$, since $B$ is diagonal and it's smallest value
is $1^2(1+1)^2$ since the sole $l=0$ term has been eliminated by enforcing $c_0 = \mean(f)$.
\begin{align}
\dfrac{\|c_{i+2} - c_{i+1}\|}{\|c_{i+1} - c_i\|} \leq 2 \lambda \|B^{-1}\| \|M^T M\|
= \dfrac{\lambda}{2}\|M^T M\| = \dfrac{\lambda}{2} \|M\|^2
\end{align}
Since convergence happens when
\[
\dfrac{\|c_{i+2} - c_{i+1}\|}{\|c_{i+1} - c_i\|} < 1
\]
convergence happens when
\begin{equation}\label{l_ineq}
\|M\| < \sqrt{2/\lambda}
\end{equation}
Recall that the smaller $\lambda$ is, the smoother the fit is.
So this places a lower limit on how smooth the fit must be.

\subsection{A quick note on choosing $\lambda$}
By definition, we have that
\[
    \| M \| = \| M^T \| \ge \dfrac{\|M^Tx\|}{\|x\|} \;\;\; \forall \; x 
\]
\[
    \| M \| \ge \dfrac{\|M^Tf\|}{\|f\|}
\]
So we can speculate that $\|M\| \approx \frac{1}{2} \dfrac{\|M^Tf\|}{\|f\|}$ will
suffice to serve the following purpose: 
Thus we can form an initial guess for a $\lambda$ such that \ref{iter_eqn} by
taking
\[
    \lambda = \dfrac{\|f\|^2}{\|M^T f\|} = \dfrac{1}{2} \dfrac{2 \cdot f^T f}{(M^T f)^T M^T f}
\]

\subsection{The algorithm}
With a given scale factor $\alpha > 1$ and convergence tolerance $\eps$,
we can form an iterative algorithm as follows:
\begin{algorithmic}
    \Require $\alpha$, $\eps$, $M$, $f$ 
    \Ensure $c$
    \State $c_p \gets 0$
    \State $c_n \gets 2 \lambda B^{-1} M^T f$
    \State $\lambda \gets \dfrac{f^T f}{(M^T f)^T M^T f}$
    \State $i \gets 1$
    \State $\Delta_n \gets \|c_n - c_p\|$
    \While{$\Delta_n \ge \eps$}
        \State $\Delta_p \gets \Delta_n$
        \State $c_p \gets c_n$
        \State $c_n \gets 2\lambda \left( B^{-1} M^T f - B^{-1} M^T M c_p\right)$
        \State $\Delta_n \gets \|c_n - c_p\|$
        \While{$\Delta_n/\Delta_p \ge 1$}
            \State $\lambda \gets \lambda / \alpha$
            \State $c_n \gets 2\lambda \left( B^{-1} M^T f - B^{-1} M^T M c_p\right)$
            \State $\Delta_n \gets \|c_n - c_p\|$
        \EndWhile
    \EndWhile
    \State $c \gets c_n$
\end{algorithmic}
Note that $\lambda$ monotonically decreases by a factor of $\alpha$ each time
the iteration~\ref{iter_eqn} fails to get closer to the true answer.
There are considerable variations available for how to shrink $\lambda$ effectively,
but the $\alpha$ shrinking outlined above should be sufficeint.

This will ultimately converge to the largest $\lambda$ satisfying condition~\ref{l_ineq}.
Thus a smoother result could be arrived at via modifying how $\alpha$ behaves.

\subsection{Implementation notes}
\subsubsection{Spherical harmonics}
Let $k = l (l+1) + m$, then the spherical harmonics have
\[
    Y_k(\theta, \phi) = 
    \begin{cases}
        \sqrt{2} \sqrt{\dfrac{2l+1}{4 \pi}} \sqrt{\dfrac{(l-|m|)!}{(l+|m|)!}} P^{|m|}_l \left(\cos(\theta)\right) \sin(|m|\phi) & \text{if } m < 0\\
        \sqrt{\dfrac{2l+1}{4 \pi}} P^{0}_l \left(\cos(\theta)\right) & \text{if } m = 0\\
        \sqrt{2} \sqrt{\dfrac{2l+1}{4 \pi}} \sqrt{\dfrac{(l-m)!}{(l+m)!}} P^{m}_l \left(\cos(\theta)\right) \cos(|m|\phi) & \text{if } m > 0\\
    \end{cases}
\]
Where $P^m_l \left(\cos(\theta) \right)$ can be formed using the following procedure
\begin{algorithmic}
    \Require $L$, $\theta$
    \State $P_0^0 \gets 0$
    \For{$l = 0,L-1$}
        \State $P_{l+1}^{l+1} \gets -(2l+1)\sin(\theta)P_l^l$
        \For{$m=0,l$}\\
            $P_{l+1}^m \gets \dfrac{(2l+1)\cos(\theta)P_l^m - (l+m)P_{l-1}^m}{l-m+1}$
        \EndFor
    \EndFor
\end{algorithmic}
Note that this must be done for each $\theta$ in the data set.


\end{document}
